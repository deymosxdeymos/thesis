\newpage
\chapter{Pendahuluan} \label{Bab I}

\section{Latar Belakang Masalah} \label{I.Latar Belakang}
Kemampuan manusia untuk berkolaborasi merupakan fondasi fundamental dalam
membangun peradaban modern. Di era globalisasi dan transformasi digital,
kolaborasi menjadi semakin kritis seiring dengan meningkatnya kompleksitas
tantangan global seperti perubahan iklim, kesehatan publik, dan inovasi
teknologi. Penelitian terbaru menunjukkan bahwa manusia memiliki dorongan kuat
untuk bekerja sama dengan orang lain, yang menjadi penggerak utama transformasi
sosial di era digital \cite{Zygmuntowski2022}.

Dalam konteks pendidikan, khususnya pembelajaran kolaboratif, kolaborasi
memainkan peran krusial dalam meningkatkan keterampilan akademik dan sosial
siswa. Pembelajaran kolaboratif melibatkan siswa yang bekerja sama secara aktif
untuk mencapai tujuan bersama, yang tidak hanya meningkatkan pemahaman
konseptual tetapi juga mengembangkan keterampilan interpersonal dan pemecahan
masalah \cite{Kotsonis2022}. Menurut penelitian tentang peran pembelajaran
kolaboratif dalam pendidikan online, kolaborasi memungkinkan siswa
mengembangkan keterampilan berpikir kritis dan komunikasi serta mengurangi
perasaan isolasi yang sering dirasakan dalam pembelajaran individu
\cite{Lu2022}. Seiring dengan berkembangnya kompleksitas dunia modern,
institusi pendidikan kini dihadapkan pada tantangan untuk mempersiapkan siswa
menghadapi lingkungan yang semakin terkoneksi dan dinamis. Keberhasilan di masa
depan tidak hanya bergantung pada penguasaan materi akademik, tetapi juga pada
kemampuan untuk berkolaborasi secara efektif dalam tim yang beragam. Hal ini
menjadikan pembelajaran kolaboratif bukan hanya sebagai metode pengajaran,
tetapi sebagai komponen vital dalam membentuk generasi yang siap menghadapi
tantangan masa depan.

Selain itu, kolaborasi merupakan salah satu keterampilan lulusan yang
fundamental dalam pendidikan tinggi dan sering menjadi atribut utama dalam
penilaian kualitas pendidikan. Kolaborasi yang efektif membantu siswa
mengembangkan kemampuan bekerja dalam tim, berpikir kritis, dan keterampilan
pemecahan masalah yang sangat relevan dengan dunia kerja. Ellis et al. (2021)
menekankan bahwa kolaborasi dalam kelompok kecil dengan pendekatan pembelajaran
mendalam, siswa dapat bekerja lebih efektif dan saling mendukung dalam memahami
materi \cite{Ellis2021}.

Meskipun memiliki banyak manfaat, pembelajaran kolaboratif juga menghadapi
tantangan signifikan, terutama dalam pembentukan kelompok yang seimbang.
Kesulitan ini sering muncul karena kurangnya informasi tentang keterampilan,
preferensi, dan dinamika pribadi siswa, terutama di awal semester ketika
pengajar belum familiar dengan profil siswa \cite{Mburasek2021}. Pengajar
biasanya harus membentuk kelompok secara manual dengan mengamati langsung
interaksi siswa, menilai keterampilan dan preferensi mereka berdasarkan
pengalaman pribadi, atau menggunakan metode pengelompokan sederhana seperti
pembagian acak. Pendekatan ini tidak hanya memakan waktu tetapi juga sering
gagal menghasilkan kelompok yang optimal. Akibatnya, terjadi berbagai masalah
seperti ketidakseimbangan keterampilan antar anggota kelompok, konflik
interpersonal, dan partisipasi yang tidak merata. Hal ini dapat menurunkan
efektivitas kolaborasi, menurunkan motivasi siswa, dan pada akhirnya
mempengaruhi kualitas hasil pembelajaran \cite{Vinella2022}. Fenomena ini
menjadi semakin menantang dalam konteks pembelajaran hybrid dan daring, di mana
interaksi tatap muka terbatas dan pengajar memiliki kesulitan lebih besar dalam
memahami dinamika kelas. Oleh karena itu, dibutuhkan sistem pembentukan tim
yang dapat mempertimbangkan berbagai faktor ini untuk meningkatkan kualitas
pembelajaran kolaboratif.

Termotivasi dari permasalahan di atas, satu solusi inovatif yang menjanjikan
datang dari Spanish National Research Council (CSIC). CSIC adalah institusi
penelitian publik terbesar di Spanyol yang berdidikasi untuk penelitian imiah
dan teknologi di berbagai bidang termasuk bidang edukasi \cite{Bernal2011}.
Mereka mengembangkan platform yang bernama EduTeams, dibuat sebagai sistem
pembagian kelompok bernama dalam lingkungan kelas secara otomatis menggunakan
bantuan kecerdasan buatan atau Aritficial Intelligence (AI) berdasarkan
keterampilan, kepribadian dan preferensi mahasiswa. Sistem ini bekerja dengan
mengumpulkan data yang diperlukan dari setiap mahasiswa di kelas seperti
tingkat keterampilan, kepribadian dan preferensi dari tugas yang diberikan.
Setelah data selesai dikumpulkan, setiap mahasiswa akan terbagi secara merata
berdasarkan data yang telah mereka isi \cite{Georgara2023}.

Saat ini tim pengembang EduTeams menyediakan \textit{Application Programming
	Interface} (API) \cite{Nawaz2022} untuk algoritma EduTeams yang bernama Edu2Com
yang akan digunakan oleh peneliti untuk mengembangkan sistem pembagian kelompok
menggunakan kecerdasan buatan. Untuk mengimplementasikan sistem pembagian
kelompok dengan kecerdasan buatan, diperlukan pengembangan dengan performa yang
baik dan dapat dijalankan oleh semua pengguna. Oleh karena itu, peran
pengembangan (\textit{development}) sangat penting untuk berlangsungnya sistem
yang berjalan dengan mulus \cite{Govea2023}. Faktor lain yang harus
dipertimbangkan untuk kelancaran proses pengembangan adalah metode
\textit{Software Development Life Cycle} (SDLC) yang akan digunakan.

Dalam proyek ini, kami mengadopsi \textit{Agile Kanban} sebagai kerangka kerja
\textit{SDLC}, yang berfokus pada visualisasi alur kerja dan pembatasan
pekerjaan-dalam-proses atau \textit{work in progress} (WIP). Pendekatan ini
sangat kontras dengan metode berbasis iterasi seperti \textit{Scrum}, yang
memaksakan progress yang telah ditentukan sebelumnya pada tim, berpotensi
menyebabkan keterlambatan dalam beradaptasi dengan perubahan yang muncul dari
hasil pengembangan dan umpan balik dari pengujian sistem. Alaidaros et al. (2021) mencatat bahwa kanban memungkinkan penyesuaian proses yang berkelanjutan, sesuai dengan kebutuhan siklus pengembangan iteratif dalam pengembangan sistem dan optimasi pengembangan. Peneliti secara khusus akan melacak metrik kunci seperti waktu siklus rata-rata untuk menyelesaikan fitur, tingkat throughput fitur baru, dan jumlah WIP. Dengan demikian, ketahanan kami dalam menanggapi kebutuhan yang berkembang seiring dengan berjalannya kebutuhan dan desain dari sistem \cite{Alaidaros2021}.


Dalam proses pengembangan, peneliti menggunakan teknik \textit{grey box
	testing}, yang menggabungkan teknik \textit{black box} (pengujian dari
perspektif pengguna, tanpa pengetahuan tentang kode internal) dengan teknik
\textit{white box} (verifikasi struktur kode dan logika). Tidak seperti
pengujian \textit{black box murni}, yang mungkin gagal menemukan kesalahan yang
berkaitan dengan pemrosesan data atau integrasi api, \textit{grey box testing}
memungkinkan pemeriksaan komponen dan alur data kunci tanpa memerlukan akses
penuh ke kode sumber yang mendasarinya \cite{Khan2012}. Khususnya, peneliti
akan menggunakan skenario \textit{grey box} untuk memvalidasi kebenaran data
yang diterima dari API edu2com, memastikan bahwa struktur data JSON diproses
dengan benar dan bahwa respons kesalahan ditangani dengan tepat. Hal ini
dicapai melalui analisis model data API edu2com dan menghasilkan kasus
pengujian yang menargetkan alur transformasi data yang kritis. Selain itu, kami
akan menggunakan \textit{grey box testing} untuk memverifikasi implementasi
algoritma pembentukan tim yang dipandu oleh AI. Dengan berfokus pada perilaku
sistem terhadap berbagai kombinasi data siswa, bertujuan untuk mengidentifikasi
potensi bias, anomali, atau perilaku tak terduga yang mungkin tidak muncul
melalui pengujian black box saja.

Penelitian ini sangat relevan dalam konteks tantangan pendidikan modern, yang
menekankan inklusi dan kesetaraan. Meskipun sistem pembentukan tim berbasis AI
memiliki potensi untuk meningkatkan keseimbangan dalam hal gender, latar
belakang, dan keterampilan, kami menyadari bahwa teknologi saja tidak dapat
menjamin hasil yang adil. Sebaliknya, kami bertujuan untuk mengeksplorasi
bagaimana sistem ini dapat digunakan secara bertanggung jawab untuk mendukung
pengambilan keputusan manusia, memberikan wawasan tentang dinamika kelompok
yang mungkin terlewatkan oleh pengajar, dan menciptakan lingkungan belajar yang
lebih inklusif.

\section{Rumusan Masalah} \label{I.Rumusan Masalah}
Dari permasalahan yang telah diidentifikasikan pada latar belakang, maka
rumusan masalah yang menjadi fokus penelitian adalah sebagai berikut:

\begin{enumerate}
	\item Bagaimana merancang sistem pembagian kelompok berbasis kecerdasan buatan yang efektif menggunakan API Edu2Com?
	\item Bagaimana hasil pengujian terhadap sistem pembagian kelompok berbasis kecerdasan buatan yang efektif menggunakan API Edu2Com?
\end{enumerate}

\section{Tujuan Penelitian} \label{I.Tujuan}
Tujuan dari penelitian ini adalah:

\begin{enumerate}
	\item Merancang dan mengembangkan sistem pembagian kelompok berbasis AI
	      menggunakan API Edu2Com dari CSIC, yang dapat mengotomatisasi proses
	      pembentukan tim berdasarkan keterampilan, kepribadian, dan preferensi siswa.
	\item Mengimplementasikan metode Agile Kanban dalam proses pengembangan
	      perangkat lunak untuk memastikan fleksibilitas dan kontinuitas perbaikan selama pengembangan.
	\item Menggunakan metode \textit{grey box testing} untuk menguji
	      fungsionalitas sistem secara menyeluruh.
\end{enumerate}

\section{Batasan Masalah}  \label{I.Batasan}
Agar penelitian ini terfokus dan dapat diselesaikan dengan efektif dalam waktu
yang ditentukan, maka ditetapkan beberapa batasan masalah sebagai berikut:

\begin{enumerate}
	\item Penelitian hanya dilakukan pada platform EduTeams.
	\item Pengembangan dashboard dibuat hanya berbasis website.
	\item Fokus pengembangan hanya pada dashboard pembagian kelompok pada sisi
	      mahasiswa dan dosen.
\end{enumerate}

\section{Manfaat Penelitian} \label{I.Manfaat}
Adapun manfaat dari penelitian ini sebagai berikut:

\begin{enumerate}
	\item Membantu dosen atau tenaga pendidik dalam membagi kelompok pada
	      mahasiswa sesuai dengan jenis kelamin, keahlian, kepribadian, dan preferensi.
	\item Mempermudah mahasiswa dalam menemukan kelompok
\end{enumerate}

\section{Sistematika Penulisan} \label{I.Sistematika}
Dalam sistematika penulisan ini dijabarkan poin-poin dari isi setiap bab.
Sistematika penulisan pada penelitian adalah sebagai berikut :

\subsection{Bab I Pendahuluan}
Pada penelitian ini, Bab I Pendahuluan akan membahas pengenalan penelitian yang
mencakup Latar Belakang Masalah, Rumusan Masalah, Tujuan Penelitian, Batasan
Masalah, Manfaat Penelitian, serta Sistematika Penulisan. Bagian ini memberikan
gambaran awal mengenai pentingnya pengembangan dashboard pembagian kelompok
berbasis \textit{Artificial Intelligence} menggunakan API Edu2Com dan
metodologi Agile Kanban sebagai solusi dalam meningkatkan efektivitas
kolaborasi mahasiswa.

\subsection{Bab II Tinjauan Pustaka}
Pada penelitian ini, Bab II Tinjauan Pustaka akan mendiskusikan landasan teori
dan kajian pustaka yang relevan, seperti konsep Sistem Informasi, Dashboard,
\textit{Artificial Intelligence}, serta algoritma dan parameter yang digunakan
pada API Edu2Com. Selain itu, akan dibahas pula metode Agile Kanban,
\textit{Unified Modeling Language} (UML), \textit{Entity Relationship Diagram}
(ERD), dan metode \textit{Grey Box Testing} yang mendukung pengembangan sistem
ini. Kajian pustaka bertujuan untuk memberikan dasar teoritis yang kuat dalam
pengembangan dan pengujian sistem.

\subsection{Bab III Metode Penelitian}
Pada penelitian ini, Bab III Metode Penelitian akan menjelaskan lebih rinci
mengenai alur penelitian, langkah-langkah penelitian, alat dan bahan yang
digunakan, serta metode pengembangan. Tahapan yang dijelaskan meliputi
perencanaan fitur, implementasi metodologi Agile Kanban, dan pengujian sistem
menggunakan \textit{Grey Box Testing}. Selain itu, Bab III juga akan membahas
pengelolaan data dan skenario pengujian
