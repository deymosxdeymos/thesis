\newpage
\chapter{HASIL DAN PEMBAHASAN} \label{Bab IV}

\section{Hasil Penelitian} \label{IV.Hasil}
Berisi hasil penelitian berdasarkan rancangan yang sudah dijelaskan pada Bab \ref{Bab III}, terutama dari Subbab \ref{III.Metode}. Bagi yang membuat alat, jelaskan alat yang jadi dalam bentuk apa. Bagi yang membuat aplikasi, jelaskan aplikasi yang jadi dalam bentuk seperti apa. Jabarkan dalam bentuk pseudocode dan dijelaskan per bagian kodenya. Gunakan gambar dan tabel sebagai alat bantu menjelaskan hasil. \par

\subsection{Implementasi Fitur}

Tabel \ref{tab:implementasi-fitur} menunjukkan daftar fitur dan menu yang telah diimplementasikan dalam sistem EduTeams menggunakan metodologi Agile Kanban. Implementasi menggunakan pendekatan iteratif dengan fokus pada fitur-fitur inti terlebih dahulu, kemudian dikembangkan ke fitur-fitur lanjutan. Status implementasi mengikuti alur kerja Kanban dengan tiga kategori: \textit{DONE} (fitur telah diimplementasikan dan diuji), \textit{IN PROGRESS} (fitur sedang dalam tahap pengembangan atau perbaikan), dan \textit{TODO} (fitur direncanakan namun belum diimplementasikan).

\begin{longtable}{|p{0.05\textwidth}|p{0.62\textwidth}|p{0.18\textwidth}|}
\caption{Implementasi Fitur Sistem EduTeams}
\label{tab:implementasi-fitur}\\
\hline
\textbf{No} & \textbf{Fitur/Menu} & \textbf{Status} \\
\hline
\endfirsthead
\hline
\textbf{No} & \textbf{Fitur/Menu} & \textbf{Status} \\
\hline
\endhead
\hline
\endfoot
\hline
\endlastfoot
\multicolumn{3}{|c|}{\textbf{A. AUTENTIKASI DAN OTORISASI}} \\
\hline
1 & Login dengan Google OAuth & \textit{DONE} \\
\hline
2 & Role-Based Access Control (RBAC) - 3 role: Mahasiswa, Dosen, Admin & \textit{DONE} \\
\hline
3 & Session Management dengan tracking IP dan user agent & \textit{DONE} \\
\hline
\multicolumn{3}{|c|}{\textbf{B. ONBOARDING PENGGUNA}} \\
\hline
4 & Pemilihan Role (Mahasiswa/Dosen) dengan auto-detection email institusi & \textit{DONE} \\
\hline
5 & Form Data Diri (NIM/NPM, nama, jenis kelamin) & \textit{DONE} \\
\hline
6 & Tes Kepribadian MBTI dengan 60 pertanyaan OJTS 1.2 & \textit{DONE} \\
\hline
7 & Sistem scoring dan klasifikasi 16 tipe MBTI (EI, SN, TF, PJ) & \textit{DONE} \\
\hline
8 & Resume Onboarding untuk melanjutkan proses yang belum selesai & \textit{DONE} \\
\hline
9 & Welcome Splash Screen untuk pengguna baru & \textit{DONE} \\
\hline
\multicolumn{3}{|c|}{\textbf{C. DASHBOARD}} \\
\hline
10 & Dashboard Mahasiswa dengan daftar kelas yang diikuti & \textit{DONE} \\
\hline
11 & Dashboard Dosen dengan statistik (total kelas, mahasiswa, tugas) & \textit{DONE} \\
\hline
12 & Navigasi Sidebar dan Top Navigation Bar & \textit{DONE} \\
\hline
\multicolumn{3}{|c|}{\textbf{D. MANAJEMEN KELAS}} \\
\hline
13 & Pembuatan Kelas Baru (nama mata kuliah, kelas, tahun, periode) & \textit{DONE} \\
\hline
14 & Join Kelas via Share Token untuk mahasiswa & \textit{DONE} \\
\hline
15 & Daftar Mahasiswa per Kelas dengan data MBTI & \textit{DONE} \\
\hline
16 & Kelola Kelas (Edit, Hapus, Arsip) & \textit{DONE} \\
\hline
17 & Search dan Filter Mahasiswa & \textit{DONE} \\
\hline
18 & Hapus Mahasiswa dari Kelas & \textit{DONE} \\
\hline
\multicolumn{3}{|c|}{\textbf{E. MANAJEMEN TUGAS}} \\
\hline
19 & Pembuatan Tugas (Assignment) dengan judul, deskripsi, tanggal mulai & \textit{DONE} \\
\hline
20 & Kelola Tugas dalam bentuk Table View dengan search dan filter & \textit{DONE} \\
\hline
21 & Edit Tugas dengan validation & \textit{DONE} \\
\hline
22 & Sistem Versioning Assignment untuk tracking perubahan & \textit{DONE} \\
\hline
23 & Deteksi Perubahan Assignment dengan flagging submission & \textit{DONE} \\
\hline
24 & Hapus dan Arsip Tugas & \textit{DONE} \\
\hline
25 & Sistem Topik per Assignment & \textit{DONE} \\
\hline
\multicolumn{3}{|c|}{\textbf{F. QUIZ DAN PREFERENSI}} \\
\hline
26 & Quiz Penilaian Skill dengan Likert scale & \textit{DONE} \\
\hline
27 & Quiz Preferensi Topik untuk assignment & \textit{DONE} \\
\hline
28 & Submission Tracking dengan status (Belum Isi, Menunggu, Berhasil) & \textit{DONE} \\
\hline
29 & Resubmission untuk assignment yang berubah & \textit{DONE} \\
\hline
\multicolumn{3}{|c|}{\textbf{G. PEMBENTUKAN KELOMPOK}} \\
\hline
30 & Integrasi Algoritma Pembentukan Kelompok (Edu2com API) & \textit{DONE} \\
\hline
31 & Algoritma Multi-Faktor (MBTI, Skill, Gender, Preferensi Topik) & \textit{DONE} \\
\hline
32 & Pembentukan kelompok berdasarkan jumlah kelompok & \textit{DONE} \\
\hline
33 & Pembentukan kelompok berdasarkan ukuran kelompok & \textit{DONE} \\
\hline
34 & Team Quality Scoring untuk setiap kelompok & \textit{DONE} \\
\hline
35 & Tampilan Hasil Kelompok untuk Dosen dan Mahasiswa & \textit{DONE} \\
\hline
36 & Reset dan Ulang Pembentukan Kelompok & \textit{DONE} \\
\hline
\multicolumn{3}{|c|}{\textbf{H. ANALITIK DAN VISUALISASI}} \\
\hline
37 & Chart Distribusi 16 Tipe MBTI & \textit{DONE} \\
\hline
38 & Chart Distribusi Skill Mahasiswa & \textit{DONE} \\
\hline
39 & Chart Distribusi Gender & \textit{DONE} \\
\hline
40 & Radar Chart Personality (EI, SN, TF, PJ) & \textit{DONE} \\
\hline
41 & Statistik Submission Rate Assignment & \textit{DONE} \\
\hline
\multicolumn{3}{|c|}{\textbf{I. PROFIL DAN PENGATURAN}} \\
\hline
42 & Halaman Profil dengan data lengkap MBTI & \textit{DONE} \\
\hline
43 & Deskripsi karakteristik 16 Tipe MBTI & \textit{DONE} \\
\hline
44 & Language Switcher (Bahasa Indonesia dan English) & \textit{DONE} \\
\hline
45 & CTA dan Navigasi pada Landing Page & \textit{DONE} \\
\hline
\multicolumn{3}{|c|}{\textbf{J. FITUR TAMBAHAN}} \\
\hline
46 & Loading States dengan Suspense dan Skeleton Components & \textit{DONE} \\
\hline
47 & Error Handling dan Error Pages & \textit{DONE} \\
\hline
48 & Responsive Design (Mobile, Tablet, Desktop) & \textit{DONE} \\
\hline
49 & Fitur Aksesibilitas Dasar (ARIA Labels pada UI Components) & \textit{DONE} \\
\hline
50 & Multi-layer Caching Strategy (Memory + Redis) & \textit{DONE} \\
\hline
\multicolumn{3}{|c|}{\textbf{K. FITUR DALAM PENGERJAAN}} \\
\hline
51 & Pre-fill Jawaban Quiz Skill dan Preferensi Sebelumnya & \textit{IN PROGRESS} \\
\hline
52 & Halaman Management untuk Mahasiswa dengan Tabs & \textit{IN PROGRESS} \\
\hline
\multicolumn{3}{|c|}{\textbf{L. FITUR YANG DIRENCANAKAN}} \\
\hline
53 & Sistem Preferensi Antar Mahasiswa (Person-to-Person Preferences) & \textit{TODO} \\
\hline
54 & Notifikasi Real-time menggunakan WebSocket/SSE & \textit{TODO} \\
\hline
55 & Notifikasi Email untuk event penting menggunakan Resend & \textit{TODO} \\
\hline
56 & Export Data Kelompok ke PDF dan Spreadsheet & \textit{TODO} \\
\hline
57 & Notification Badge pada Management Page & \textit{TODO} \\
\hline
58 & Advanced Settings untuk Bobot Pembagian Kelompok & \textit{TODO} \\
\hline
59 & Legend dan Sort untuk MBTI Chart & \textit{TODO} \\
\hline
\end{longtable}

\noindent
Dari total 59 fitur yang diidentifikasi, 50 fitur (84.7\%) telah selesai diimplementasikan dan diuji, 2 fitur (3.4\%) sedang dalam tahap pengerjaan berupa implementasi fitur tambahan untuk mahasiswa, serta 7 fitur (11.9\%) direncanakan untuk pengembangan selanjutnya. Tingkat kelengkapan implementasi yang tinggi ini menunjukkan bahwa sistem EduTeams telah mencapai tahap maturity yang baik untuk fitur-fitur inti dan siap untuk digunakan dalam lingkungan produksi dengan rencana pengembangan berkelanjutan untuk fitur-fitur tambahan.

Contoh implementasi kode dapat ditulis menggunakan \verb|\begin{lstlisting}|. Contoh kode dapat dilihat pada Kode \ref{code:4.contoh}. \par
% Menulis blok kode
\begin{lstlisting}[caption={Akuisisi Gambar}, label={code:4.contoh}]
def process_dataset(dataset_path):
	image_files = glob(os.path.join(dataset_path, '*.png'))
	image_files.sort()
	for image_file in image_files:
		frame = cv2.imread(image_file)
		if frame is None:
			continue
		frame_rgb = cv2.cvtColor(frame, cv2.COLOR_BGR2RGB)
		cv2.imshow('Frame', frame)
		if cv2.waitKey(1) & 0xFF == ord('q'):
			break
	cv2.destroyAllWindows()
def main():
	datasets = get_all_dataset_folders(DATASET_ROOT)
	for dataset in datasets:
		process_dataset(dataset)
		print("print string")
\end{lstlisting}

\section{Hasil Pengujian} \label{IV.Hasil_Uji}
Berikan hasil pengujian berdasarkan rancangan \& skenario yang sudah direncanakan sebelumnya pada Subbab \ref{III.Rancang_Uji}. \par

\begin{longtable}{|c|c|c|c|c|c|c|c|c|}
	\caption{Data \textit{dummy} Pengujian}
	\label{table:4.dummy}\\
	\hline
	\multirow{2}{*}{\textbf{Subjek}} & \multicolumn{7}{|c|}{\textbf{Hasil Prediksi (BPM)}} & \multirow{2}{*}{\textbf{GT}} \\ \cline{2-8}
	& \textbf{F} & \textbf{NA} & \textbf{NO} & \textbf{RC} & \textbf{LC} & \textbf{M} & \textbf{C} & \\
	\hline
	\endfirsthead
	\hline
	\multirow{2}{*}{\textbf{Subjek}} & \multicolumn{7}{|c|}{\textbf{Hasil Prediksi (BPM)}} & \multirow{2}{*}{\textbf{GT}} \\ \cline{2-8}
	& \textbf{F} & \textbf{NA} & \textbf{NO} & \textbf{RC} & \textbf{LC} & \textbf{M} & \textbf{C} & \\
	\hline
	\endhead
	\hline
	\endfoot
	\hline
	\endlastfoot
	1 & 68 & 69 & 68 & 70 & 68 & 71 & 69 & 68 \\
	\hline
	2 & 69 & 69 & 68 & 70 & 68 & 71 & 69 & 69 \\
	\hline
	3 & 70 & 70 & 69 & 71 & 68 & 73 & 69 & 70\\
	\hline
	4 & 71 & 70 & 70 & 72 & 69 & 73 & 70 & 71 \\
	\hline
	5 & 72 & 72 & 70 & 72 & 70 & 74 & 70 & 72 \\
\end{longtable}

\begin{figure}[H]
	\centering
	\includegraphics[width=0.7\textwidth]{figure/zeta.png}
	\caption{Contoh Graf Pengujian}
	\label{fig:4.graf}
\end{figure}

\section{Analisis Hasil Penelitian} \label{IV.Analisis}
Berikan analisis hasil penelitian \& pengujian, berupa data yang didapatkan dari penelitian \& pengujian Tugas Akhir yang sudah anda kerjakan. Gunakan gambar dan tabel sebagai alat bantu menjelaskan analisis hasil. Data luaran penelitian yang dapat dianalisis berupa: \par
\begin{enumerate}[noitemsep]
	\item Hasil pengujian
	\item Hasil kuesioner
	\item Aplikasi yang dikembangkan
\end{enumerate}
Analisis dapat membandingkan dengan hasil penelitian sebelumnya yang memiliki kemiripan topik. \par
