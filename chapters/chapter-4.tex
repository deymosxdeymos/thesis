\newpage
\chapter{HASIL DAN PEMBAHASAN} \label{Bab IV}

\section{Hasil Penelitian} \label{IV.Hasil}
Berisi hasil penelitian berdasarkan rancangan yang sudah dijelaskan pada Bab \nameref{Bab III}, terutama dari Subbab \ref{III.Metode}. Bagi yang membuat alat, jelaskan alat yang jadi dalam bentuk apa. Bagi yang membuat aplikasi, jelaskan aplikasi yang jadi dalam bentuk seperti apa. Jabarkan dalam bentuk pseudocode dan dijelaskan per bagian kodenya. Gunakan gambar dan tabel sebagai alat bantu menjelaskan hasil. \par

\subsection{Implementasi Fitur}

Tabel \ref{tab:implementasi-fitur} menunjukkan daftar fitur dan menu yang telah diimplementasikan dalam sistem EduTeams menggunakan metodologi Agile Kanban. Implementasi menggunakan pendekatan iteratif dengan fokus pada fitur-fitur inti terlebih dahulu, kemudian dikembangkan ke fitur-fitur lanjutan. Status implementasi mengikuti alur kerja Kanban dengan tiga kategori: \textit{DONE} (fitur telah diimplementasikan dan diuji), \textit{IN PROGRESS} (fitur sedang dalam tahap pengembangan atau perbaikan), dan \textit{TODO} (fitur direncanakan namun belum diimplementasikan).

\subsubsection*{Fitur Dosen (Dosen Features)}

\begin{longtable}{|p{0.05\textwidth}|p{0.62\textwidth}|p{0.18\textwidth}|}
	\caption{Implementasi Fitur Sistem EduTeams}
	\label{tab:implementasi-fitur}                                                                              \\
	\hline
	\textbf{No} & \textbf{Fitur/Menu}                                                    & \textbf{Status}      \\
	\hline
	\endfirsthead
	\hline
	\textbf{No} & \textbf{Fitur/Menu}                                                    & \textbf{Status}      \\
	\hline
	\endhead
	\hline
	\endfoot
	\hline
	\endlastfoot
	\multicolumn{3}{|c|}{\textbf{A. AUTENTIKASI DAN OTORISASI}}                                                 \\
	\hline
	1           & Login dengan Google OAuth                                              & \textit{DONE}        \\
	\hline
	2           & Role-Based Access Control (RBAC) - 3 role: Mahasiswa, Dosen, Admin     & \textit{DONE}        \\
	\hline
	3           & Session Management dengan tracking IP dan user agent                   & \textit{DONE}        \\
	\hline
	\multicolumn{3}{|c|}{\textbf{B. ONBOARDING PENGGUNA}}                                                       \\
	\hline
	4           & Pemilihan Role (Mahasiswa/Dosen) dengan auto-detection email institusi & \textit{DONE}        \\
	\hline
	5           & Form Data Diri (NIM/NPM, nama, jenis kelamin)                          & \textit{DONE}        \\
	\hline
	6           & Tes Kepribadian MBTI dengan 60 pertanyaan OJTS 1.2                     & \textit{DONE}        \\
	\hline
	7           & Sistem scoring dan klasifikasi 16 tipe MBTI (EI, SN, TF, PJ)           & \textit{DONE}        \\
	\hline
	8           & Resume Onboarding untuk melanjutkan proses yang belum selesai          & \textit{DONE}        \\
	\hline
	9           & Welcome Splash Screen untuk pengguna baru                              & \textit{DONE}        \\
	\hline
	\multicolumn{3}{|c|}{\textbf{C. DASHBOARD}}                                                                 \\
	\hline
	10          & Dashboard Mahasiswa dengan daftar kelas yang diikuti                   & \textit{DONE}        \\
	\hline
	11          & Dashboard Dosen dengan statistik (total kelas, mahasiswa, tugas)       & \textit{DONE}        \\
	\hline
	12          & Navigasi Sidebar dan Top Navigation Bar                                & \textit{DONE}        \\
	\hline
	\multicolumn{3}{|c|}{\textbf{D. MANAJEMEN KELAS}}                                                           \\
	\hline
	13          & Pembuatan Kelas Baru (nama mata kuliah, kelas, tahun, periode)         & \textit{DONE}        \\
	\hline
	14          & Join Kelas via Share Token untuk mahasiswa                             & \textit{DONE}        \\
	\hline
	15          & Daftar Mahasiswa per Kelas dengan data MBTI                            & \textit{DONE}        \\
	\hline
	16          & Kelola Kelas (Edit, Hapus, Arsip)                                      & \textit{DONE}        \\
	\hline
	17          & Search dan Filter Mahasiswa                                            & \textit{DONE}        \\
	\hline
	18          & Hapus Mahasiswa dari Kelas                                             & \textit{DONE}        \\
	\hline
	\multicolumn{3}{|c|}{\textbf{E. MANAJEMEN TUGAS}}                                                           \\
	\hline
	19          & Pembuatan Tugas (Assignment) dengan judul, deskripsi, tanggal mulai    & \textit{DONE}        \\
	\hline
	20          & Kelola Tugas dalam bentuk Table View dengan search dan filter          & \textit{DONE}        \\
	\hline
	21          & Edit Tugas dengan validation                                           & \textit{DONE}        \\
	\hline
	22          & Sistem Versioning Assignment untuk tracking perubahan                  & \textit{DONE}        \\
	\hline
	23          & Deteksi Perubahan Assignment dengan flagging submission                & \textit{DONE}        \\
	\hline
	24          & Hapus dan Arsip Tugas                                                  & \textit{DONE}        \\
	\hline
	25          & Sistem Topik per Assignment                                            & \textit{DONE}        \\
	\hline
	\multicolumn{3}{|c|}{\textbf{F. QUIZ DAN PREFERENSI}}                                                       \\
	\hline
	26          & Quiz Penilaian Skill dengan Likert scale                               & \textit{DONE}        \\
	\hline
	27          & Quiz Preferensi Topik untuk assignment                                 & \textit{DONE}        \\
	\hline
	28          & Submission Tracking dengan status (Belum Isi, Menunggu, Berhasil)      & \textit{DONE}        \\
	\hline
	29          & Resubmission untuk assignment yang berubah                             & \textit{DONE}        \\
	\hline
	\multicolumn{3}{|c|}{\textbf{G. PEMBENTUKAN KELOMPOK}}                                                      \\
	\hline
	30          & Integrasi Algoritma Pembentukan Kelompok (Edu2com API)                 & \textit{DONE}        \\
	\hline
	31          & Algoritma Multi-Faktor (MBTI, Skill, Gender, Preferensi Topik)         & \textit{DONE}        \\
	\hline
	32          & Pembentukan kelompok berdasarkan jumlah kelompok                       & \textit{DONE}        \\
	\hline
	33          & Pembentukan kelompok berdasarkan ukuran kelompok                       & \textit{DONE}        \\
	\hline
	34          & Team Quality Scoring untuk setiap kelompok                             & \textit{DONE}        \\
	\hline
	35          & Tampilan Hasil Kelompok untuk Dosen dan Mahasiswa                      & \textit{DONE}        \\
	\hline
	36          & Reset dan Ulang Pembentukan Kelompok                                   & \textit{DONE}        \\
	\hline
	\multicolumn{3}{|c|}{\textbf{H. ANALITIK DAN VISUALISASI}}                                                  \\
	\hline
	37          & Chart Distribusi 16 Tipe MBTI                                          & \textit{DONE}        \\
	\hline
	38          & Chart Distribusi Skill Mahasiswa                                       & \textit{DONE}        \\
	\hline
	39          & Chart Distribusi Gender                                                & \textit{DONE}        \\
	\hline
	40          & Radar Chart Personality (EI, SN, TF, PJ)                               & \textit{DONE}        \\
	\hline
	41          & Statistik Submission Rate Assignment                                   & \textit{DONE}        \\
	\hline
	\multicolumn{3}{|c|}{\textbf{I. PROFIL DAN PENGATURAN}}                                                     \\
	\hline
	42          & Halaman Profil dengan data lengkap MBTI                                & \textit{DONE}        \\
	\hline
	43          & Deskripsi karakteristik 16 Tipe MBTI                                   & \textit{DONE}        \\
	\hline
	44          & Language Switcher (Bahasa Indonesia dan English)                       & \textit{DONE}        \\
	\hline
	45          & CTA dan Navigasi pada Landing Page                                     & \textit{DONE}        \\
	\hline
	\multicolumn{3}{|c|}{\textbf{J. FITUR TAMBAHAN}}                                                            \\
	\hline
	46          & Loading States dengan Suspense dan Skeleton Components                 & \textit{DONE}        \\
	\hline
	47          & Error Handling dan Error Pages                                         & \textit{DONE}        \\
	\hline
	48          & Responsive Design (Mobile, Tablet, Desktop)                            & \textit{DONE}        \\
	\hline
	49          & Fitur Aksesibilitas Dasar (ARIA Labels pada UI Components)             & \textit{DONE}        \\
	\hline
	50          & Multi-layer Caching Strategy (Memory + Redis)                          & \textit{DONE}        \\
	\hline
	\multicolumn{3}{|c|}{\textbf{K. FITUR DALAM PENGERJAAN}}                                                    \\
	\hline
	51          & Pre-fill Jawaban Quiz Skill dan Preferensi Sebelumnya                  & \textit{IN PROGRESS} \\
	\hline
	52          & Halaman Management untuk Mahasiswa dengan Tabs                         & \textit{IN PROGRESS} \\
	\hline
	\multicolumn{3}{|c|}{\textbf{L. FITUR YANG DIRENCANAKAN}}                                                   \\
	\hline
	53          & Sistem Preferensi Antar Mahasiswa (Person-to-Person Preferences)       & \textit{TODO}        \\
	\hline
	54          & Notifikasi Real-time menggunakan WebSocket/SSE                         & \textit{TODO}        \\
	\hline
	55          & Notifikasi Email untuk event penting menggunakan Resend                & \textit{TODO}        \\
	\hline
	56          & Export Data Kelompok ke PDF dan Spreadsheet                            & \textit{TODO}        \\
	\hline
	57          & Notification Badge pada Management Page                                & \textit{TODO}        \\
	\hline
	58          & Advanced Settings untuk Bobot Pembagian Kelompok                       & \textit{TODO}        \\
	\hline
	59          & Legend dan Sort untuk MBTI Chart                                       & \textit{TODO}        \\
	\hline
\end{longtable}

\noindent
Dari total 59 fitur yang diidentifikasi, 50 fitur (84.7\%) telah selesai diimplementasikan dan diuji, 2 fitur (3.4\%) sedang dalam tahap pengerjaan berupa implementasi fitur tambahan untuk mahasiswa, serta 7 fitur (11.9\%) direncanakan untuk pengembangan selanjutnya. Tingkat kelengkapan implementasi yang tinggi ini menunjukkan bahwa sistem EduTeams telah mencapai tahap maturity yang baik untuk fitur-fitur inti dan siap untuk digunakan dalam lingkungan produksi dengan rencana pengembangan berkelanjutan untuk fitur-fitur tambahan.

\subsubsection{Halaman Landing Page}

Halaman landing page merupakan tampilan awal yang dilihat pengunjung saat mengakses sistem EquiTeam. Halaman ini menyediakan informasi menyeluruh tentang sistem, meliputi penjelasan konsep, identifikasi masalah, solusi yang ditawarkan, dan manfaat bagi pengguna. Landing page menyediakan tombol login Google OAuth untuk autentikasi pengguna. Landing page mengimplementasikan tata letak responsif yang dapat menyesuaikan dengan berbagai ukuran layar. Gambar \ref{fig:4.landing-page} menunjukkan tampilan halaman landing page sistem EquiTeam.

\begin{figure}[H]
	\centering
	\includegraphics[width=0.9\textwidth]{figure/chapter-4/page/landing-page-equiteam.png}
	\caption{Tampilan Halaman Landing Page Sistem EquiTeam}
	\label{fig:4.landing-page}
\end{figure}

\subsubsection*{Fitur Dosen}

Bagian ini mendokumentasikan fitur-fitur yang tersedia untuk dosen di sistem EquiTeam. Fitur-fitur ini dirancang untuk memfasilitasi manajemen kelas, pembuatan tugas, visualisasi analitik, dan koordinasi pembentukan kelompok berbasis AI.

\subsubsection{Halaman Onboarding Data Diri}

Halaman onboarding data diri merupakan bagian kedua dari alur onboarding sistem EquiTeam. Pengguna mengisi informasi pribadi sesuai dengan peran yang dipilih (Dosen atau Mahasiswa), termasuk nama lengkap, nomor identitas, dan jenis kelamin. Data ini digunakan untuk identifikasi pengguna dan pembentukan kelompok yang seimbang secara demografis.

\begin{figure}[H]
	\centering
	\includegraphics[width=0.9\textwidth]{figure/chapter-4/page/onboarding-data-diri-dosen.png}
	\caption{Halaman Onboarding Data Diri untuk Peran Dosen}
	\label{fig:4.onboarding-data-diri}
\end{figure}

\subsubsection{Dashboard Dosen - Status Kosong}

Setelah menyelesaikan proses onboarding, pengguna dengan peran Dosen diarahkan ke halaman dashboard. Dashboard ini merupakan halaman utama yang menampilkan ringkasan statistik kelas yang dikelola. Ketika Dosen belum membuat kelas apapun, dashboard menampilkan pesan "Anda belum membuat kelas" dengan tombol "Buat Kelas Baru" untuk memulai.

\begin{figure}[H]
	\centering
	\includegraphics[width=0.9\textwidth]{figure/chapter-4/page/dashboard-empty-state.png}
	\caption{Tampilan Dashboard Dosen dalam Status Kosong}
	\label{fig:4.dashboard-empty}
\end{figure}

\subsubsection{Dialog Pembuatan Kelas}

Fitur pembuatan kelas memungkinkan Dosen untuk membuat kelas baru dengan mengisi informasi yang diperlukan: Nama Mata Kuliah, Kelas (sesi/kelompok), dan Periode Akademik. Ketika Dosen mengklik tombol "Buat Kelas Baru", sistem menampilkan dialog form dengan field-field tersebut.

\begin{figure}[H]
	\centering
	\includegraphics[width=0.9\textwidth]{figure/chapter-4/page/create-class-dialog-empty.png}
	\caption{Dialog Pembuatan Kelas - Status Kosong}
	\label{fig:4.create-class-empty}
\end{figure}

Setelah dosen mengisi semua field dengan data kelas, dialog menampilkan data yang telah dimasukkan dan siap untuk disimpan.

\begin{figure}[H]
	\centering
	\includegraphics[width=0.9\textwidth]{figure/chapter-4/page/create-class-dialog-filled.png}
	\caption{Dialog Pembuatan Kelas - Data Terisi}
	\label{fig:4.create-class-filled}
\end{figure}

\subsubsection{Dashboard Dosen - Dengan Kelas}

Setelah Dosen berhasil membuat kelas, dashboard diperbarui untuk menampilkan kelas yang telah dibuat. Halaman dashboard sekarang menampilkan kartu kelas yang berisi informasi nama mata kuliah, sesi kelas, jumlah mahasiswa yang terdaftar, dan quick action buttons untuk mengelola kelas. Statistik di bagian atas juga tetap ditampilkan untuk memberikan ringkasan aktivitas. Gambar \ref{fig:4.dashboard-with-class} menunjukkan tampilan dashboard setelah kelas berhasil dibuat.

\begin{figure}[H]
	\centering
	\includegraphics[width=0.9\textwidth]{figure/chapter-4/page/dashboard-with-class.png}
	\caption{Tampilan Dashboard Dosen dengan Kelas yang Telah Dibuat}
	\label{fig:4.dashboard-with-class}
\end{figure}

\subsubsection{Halaman Detail Kelas}

Ketika Dosen mengklik pada kelas yang telah dibuat, sistem menampilkan halaman detail kelas dengan informasi lengkap dan opsi manajemen. Halaman ini menampilkan informasi kelas di bagian atas, daftar tugas di bagian tengah, dan daftar mahasiswa di bagian bawah (semua kosong pada awalnya).

\begin{figure}[H]
	\centering
	\includegraphics[width=0.9\textwidth]{figure/chapter-4/page/class-details-page.png}
	\caption{Halaman Detail Kelas - Status Kosong}
	\label{fig:4.class-details}
\end{figure}

\subsubsection{Dialog Pembagian Kelas}

Untuk memungkinkan mahasiswa mendaftar ke kelas, Dosen dapat membagikan kelas dengan mengklik tombol "Bagikan Kelas". Sistem menampilkan dialog yang berisi URL kelas dan token pembagian yang dapat dibagikan kepada mahasiswa untuk pendaftaran otomatis.

\begin{figure}[H]
	\centering
	\includegraphics[width=0.9\textwidth]{figure/chapter-4/page/share-class-dialog.png}
	\caption{Dialog Pembagian Kelas dengan Token dan URL}
	\label{fig:4.share-class}
\end{figure}

\subsubsection{Dialog Pembuatan Tugas}

Fitur pembuatan tugas memungkinkan Dosen untuk membuat assignment dengan mengisi informasi penting: Nama Tugas, Deskripsi, Spesifikasi Keahlian (skills yang dibutuhkan), dan Preferensi Topik. Data ini akan digunakan untuk pengumpulan respons mahasiswa dan algoritma pembentukan kelompok.

\begin{figure}[H]
	\centering
	\includegraphics[width=0.9\textwidth]{figure/chapter-4/page/create-assignment-dialog-empty.png}
	\caption{Dialog Pembuatan Tugas - Status Kosong}
	\label{fig:4.create-assignment-empty}
\end{figure}

Setelah dosen mengisi semua field dengan data tugas, dialog menampilkan data yang telah dimasukkan dan siap untuk disimpan.

\begin{figure}[H]
	\centering
	\includegraphics[width=0.9\textwidth]{figure/chapter-4/page/create-assignment-dialog-filled.png}
	\caption{Dialog Pembuatan Tugas - Data Terisi}
	\label{fig:4.create-assignment-filled}
\end{figure}

\subsubsection{Halaman Detail Kelas dengan Tugas}

Setelah Dosen membuat tugas, halaman detail kelas diperbarui untuk menampilkan tugas yang telah dibuat dalam bentuk daftar dengan informasi status pengumpulan respons kuesioner mahasiswa.

\begin{figure}[H]
	\centering
	\includegraphics[width=0.9\textwidth]{figure/chapter-4/page/class-with-assignment.png}
	\caption{Halaman Detail Kelas dengan Tugas yang Telah Dibuat}
	\label{fig:4.class-with-assignment}
\end{figure}

\subsubsection{Halaman Analitik Tugas}

Ketika Dosen mengklik pada tugas, sistem menampilkan halaman detail tugas dengan berbagai visualisasi data analitik yang komprehensif, meliputi: (1) Distribusi tipe kepribadian MBTI mahasiswa dengan karakter unik setiap tipe, (2) Distribusi skill mahasiswa, (3) Preferensi topik mahasiswa, dan (4) Distribusi gender untuk memastikan representasi seimbang.

\begin{figure}[H]
	\centering
	\includegraphics[width=0.9\textwidth]{figure/chapter-4/page/assignment-details-analytics.png}
	\caption{Halaman Analitik Tugas dengan Visualisasi Data Komprehensif}
	\label{fig:4.assignment-analytics}
\end{figure}

\subsubsection*{Fitur Mahasiswa}

Bagian ini mendokumentasikan fitur-fitur yang tersedia untuk mahasiswa di sistem EquiTeam. Fitur-fitur ini dirancang untuk memfasilitasi partisipasi mahasiswa dalam proses onboarding, pendaftaran kelas, pengerjaan assignment, pengisian kuesioner, dan pembentukan kelompok berbasis AI.

\subsubsection{Halaman Onboarding Data Diri - Mahasiswa}

Mahasiswa harus menyelesaikan halaman onboarding data diri sebagai bagian dari proses registrasi awal, mengisi informasi pribadi: nama lengkap, nomor induk mahasiswa (NIM), dan jenis kelamin. Form ini memiliki tampilan yang sama dengan form dosen namun disesuaikan untuk konteks mahasiswa.

\begin{figure}[H]
	\centering
	\includegraphics[width=0.9\textwidth]{figure/chapter-4/page/mahasiswa-onboarding-data-diri.png}
	\caption{Halaman Onboarding Data Diri untuk Mahasiswa - Status Kosong}
	\label{fig:4.mahasiswa-data-diri}
\end{figure}

Setelah mahasiswa mengisi semua field dengan data pribadinya, halaman menampilkan data yang telah dimasukkan dan data ini akan digunakan untuk identifikasi mahasiswa dan proses pembentukan kelompok.

\begin{figure}[H]
	\centering
	\includegraphics[width=0.9\textwidth]{figure/chapter-4/page/mahasiswa-onboarding-data-diri-filled.png}
	\caption{Halaman Onboarding Data Diri untuk Mahasiswa - Data Terisi}
	\label{fig:4.mahasiswa-data-diri-filled}
\end{figure}

\subsubsection{Tes Kepribadian MBTI}

Setelah menyelesaikan pengisian data pribadi, mahasiswa diwajibkan menyelesaikan tes kepribadian MBTI (Myers-Briggs Type Indicator) sebagai bagian integral dari proses onboarding. Data kepribadian ini akan digunakan untuk membentuk kelompok yang seimbang secara dinamis. Sebelum memulai tes, sistem menampilkan modal instruksi dengan penjelasan tentang skala Likert 5-poin yang digunakan.

\begin{figure}[H]
	\centering
	\includegraphics[width=0.9\textwidth]{figure/chapter-4/page/mahasiswa-onboarding-mbti-instructions.png}
	\caption{Modal Instruksi Tes Kepribadian MBTI}
	\label{fig:4.mahasiswa-mbti-instructions}
\end{figure}

Tes terdiri dari 42 pertanyaan yang dibagi menjadi 7 halaman (6 pertanyaan per halaman), mengidentifikasi empat dimensi kepribadian MBTI: Ekstroversion-Introversion (EI), Sensing-Intuition (SN), Thinking-Feeling (TF), dan Perception-Judging (PJ). Setiap pertanyaan menampilkan pernyataan dengan respons pada skala 5-poin.

\begin{figure}[H]
	\centering
	\includegraphics[width=0.9\textwidth]{figure/chapter-4/page/mahasiswa-onboarding-mbti-test-page1.png}
	\caption{Halaman Pertama Tes Kepribadian MBTI (Halaman 1 dari 7)}
	\label{fig:4.mahasiswa-mbti-page1}
\end{figure}

Ketika mahasiswa mengisi respons dalam tes, sistem secara real-time menampilkan pilihan yang telah dipilih dan menyimpan respons secara dinamis.

\begin{figure}[H]
	\centering
	\includegraphics[width=0.9\textwidth]{figure/chapter-4/page/mahasiswa-onboarding-mbti-test-filled.png}
	\caption{Tes Kepribadian MBTI dengan Respons yang Telah Dipilih}
	\label{fig:4.mahasiswa-mbti-filled}
\end{figure}

\subsubsection{Hasil Tes Kepribadian MBTI}

Setelah menyelesaikan seluruh pertanyaan tes kepribadian MBTI, sistem menganalisis respons mahasiswa dan menghasilkan hasil kepribadian. Hasil tes ditampilkan dengan tipe kepribadian MBTI yang diperoleh, penjelasan, dan persentase skor untuk setiap dimensi MBTI (EI, SN, TF, PJ) untuk membantu mahasiswa memahami profil kepribadian mereka.

\begin{figure}[H]
	\centering
	\includegraphics[width=0.9\textwidth]{figure/chapter-4/page/mahasiswa-mbti-result.png}
	\caption{Halaman Hasil Tes Kepribadian MBTI}
	\label{fig:4.mahasiswa-mbti-result}
\end{figure}

Sistem juga menyediakan dialog modal yang menampilkan grid lengkap dari 16 tipe kepribadian MBTI dengan ikon karakter unik untuk setiap tipe, memungkinkan mahasiswa untuk menjelajahi deskripsi dan karakteristik dari berbagai tipe kepribadian.

\begin{figure}[H]
	\centering
	\includegraphics[width=0.9\textwidth]{figure/chapter-4/page/mahasiswa-mbti-distribution.png}
	\caption{Modal Dialog Distribusi 16 Tipe Kepribadian MBTI}
	\label{fig:4.mahasiswa-mbti-distribution}
\end{figure}

\subsubsection{Dashboard Mahasiswa}

Setelah menyelesaikan proses onboarding dan tes kepribadian MBTI, mahasiswa diarahkan ke halaman dashboard yang menampilkan daftar kelas yang telah diikuti. Ketika mahasiswa pertama kali masuk dan belum mengikuti kelas apapun, halaman menampilkan pesan "Belum ada kelas yang diikuti" dengan tombol "Masuk Kelas" untuk memulai bergabung.

\begin{figure}[H]
	\centering
	\includegraphics[width=0.9\textwidth]{figure/chapter-4/page/mahasiswa-dashboard.png}
	\caption{Tampilan Dashboard Mahasiswa - Status Kosong}
	\label{fig:4.mahasiswa-dashboard}
\end{figure}

\subsubsection{Dialog Pendaftaran Kelas}

Untuk bergabung dengan kelas, mahasiswa dapat menggunakan kode atau token yang dibagikan oleh dosen. Sistem menyediakan dialog untuk memasukkan kode kelas yang memungkinkan mahasiswa terdaftar otomatis di kelas tersebut.

\begin{figure}[H]
	\centering
	\includegraphics[width=0.9\textwidth]{figure/chapter-4/page/mahasiswa-join-class-dialog.png}
	\caption{Dialog Pendaftaran Kelas - Status Kosong}
	\label{fig:4.mahasiswa-join-class}
\end{figure}

Ketika mahasiswa memasukkan token kelas yang valid, tombol "Masuk" menjadi aktif dan mahasiswa dapat melanjutkan proses pendaftaran. Mahasiswa akan otomatis terdaftar di kelas dan dapat melihat daftar assignment serta informasi kelas lainnya.

\begin{figure}[H]
	\centering
	\includegraphics[width=0.9\textwidth]{figure/chapter-4/page/mahasiswa-join-class-dialog-filled.png}
	\caption{Dialog Pendaftaran Kelas - Token Terisi}
	\label{fig:4.mahasiswa-join-class-filled}
\end{figure}

\subsubsection{Dashboard Mahasiswa dengan Kelas}

Setelah mahasiswa berhasil bergabung dengan kelas, dashboard diperbarui untuk menampilkan kelas yang telah diikuti. Kartu kelas menampilkan informasi lengkap meliputi kode kelas, jumlah mahasiswa, nama mata kuliah, nama dosen, dan periode akademik.

\begin{figure}[H]
	\centering
	\includegraphics[width=0.9\textwidth]{figure/chapter-4/page/mahasiswa-dashboard-with-class.png}
	\caption{Tampilan Dashboard Mahasiswa dengan Kelas yang Telah Diikuti}
	\label{fig:4.mahasiswa-dashboard-with-class}
\end{figure}

\subsubsection{Halaman Detail Kelas dari Perspektif Mahasiswa}

Ketika mahasiswa mengklik pada kelas yang telah diikuti, sistem menampilkan halaman detail kelas dengan daftar tugas (assignment) yang dibuat dosen beserta status pengumpulan respons mahasiswa. Di sebelah kanan, halaman menampilkan daftar mahasiswa yang tergabung dengan informasi tipe kepribadian MBTI mereka.

\begin{figure}[H]
	\centering
	\includegraphics[width=0.9\textwidth]{figure/chapter-4/page/mahasiswa-class-view.png}
	\caption{Halaman Detail Kelas dan Daftar Tugas dari Perspektif Mahasiswa}
	\label{fig:4.mahasiswa-class-view}
\end{figure}

\subsubsection{Tes Keahlian (Skills Assessment)}

Untuk setiap tugas, mahasiswa harus mengisi kuesioner tentang keahlian dan preferensi topik mereka. Sistem dimulai dengan menampilkan modal instruksi yang menjelaskan skala penilaian keahlian dengan lima tingkat: Pemula, Pemula Lanjut, Kompeten, Mahir, dan Jago Banget, dengan ikon emoji unik untuk setiap tingkat.

\begin{figure}[H]
	\centering
	\includegraphics[width=0.9\textwidth]{figure/chapter-4/page/mahasiswa-assignment-quiz.png}
	\caption{Modal Instruksi Tes Keahlian dengan Lima Tingkat Kemahiran}
	\label{fig:4.mahasiswa-assignment-quiz}
\end{figure}

Halaman tes keahlian menampilkan pertanyaan-pertanyaan yang menilai kemampuan mahasiswa di berbagai teknologi atau topik yang relevan dengan tugas. Mahasiswa harus jujur menilai kemampuan mereka karena data ini akan digunakan untuk pembentukan kelompok yang seimbang.

\begin{figure}[H]
	\centering
	\includegraphics[width=0.9\textwidth]{figure/chapter-4/page/mahasiswa-assignment-quiz-questions.png}
	\caption{Halaman Tes Keahlian Mahasiswa untuk Berbagai Teknologi}
	\label{fig:4.mahasiswa-assignment-quiz-questions}
\end{figure}

\subsubsection{Tes Preferensi Topik}

Setelah mengisi tes keahlian, mahasiswa juga harus mengisi tes preferensi topik untuk menunjukkan minat mereka terhadap berbagai topik yang tersedia dalam tugas. Tes ini menggunakan skala Likert lima poin dari "Sangat Tidak Tertarik" hingga "Sangat Tertarik". Data preferensi topik ini akan membantu algoritma pembentukan kelompok mencocokkan mahasiswa dengan topik sesuai minat mereka.

\begin{figure}[H]
	\centering
	\includegraphics[width=0.9\textwidth]{figure/chapter-4/page/mahasiswa-assignment-topic-preference.png}
	\caption{Halaman Preferensi Topik untuk Menilai Minat Mahasiswa}
	\label{fig:4.mahasiswa-assignment-topic-preference}
\end{figure}

\subsubsection{Status Menunggu Pembagian Kelompok}

Setelah mahasiswa menyelesaikan pengisian kuesioner keahlian dan preferensi topik, sistem menampilkan halaman status yang menunjukkan bahwa respons telah dikumpulkan dan sistem sedang menunggu dosen untuk membentuk kelompok. Halaman ini menampilkan ikon hourglass dengan pesan "Menunggu pembagian kelompok!" untuk menginformasikan status proses.

\begin{figure}[H]
	\centering
	\includegraphics[width=0.9\textwidth]{figure/chapter-4/page/mahasiswa-assignment-waiting-team.png}
	\caption{Halaman Status Menunggu Pembagian Kelompok oleh Dosen}
	\label{fig:4.mahasiswa-assignment-waiting-team}
\end{figure}

\subsubsection{Halaman Profil Mahasiswa}

Mahasiswa dapat melihat profil lengkap mereka yang menampilkan hasil tes kepribadian MBTI yang telah diselesaikan selama onboarding. Halaman profil menampilkan tipe kepribadian beserta penjelasan karakteristik dengan visualisasi bar chart yang menampilkan persentase skor untuk setiap dimensi MBTI (E/I, N/S, F/T, J/P).

\begin{figure}[H]
	\centering
	\includegraphics[width=0.9\textwidth]{figure/chapter-4/page/mahasiswa-profile.png}
	\caption{Halaman Profil Mahasiswa dengan Visualisasi Bar Chart Tipe Kepribadian MBTI}
	\label{fig:4.mahasiswa-profile}
\end{figure}

Sistem juga menyediakan visualisasi alternatif menggunakan grafik radar yang menampilkan profil kepribadian dalam bentuk diagram bintang dengan empat sumbu yang mewakili dimensi MBTI, memberikan perspektif berbeda tentang profil kepribadian.

\begin{figure}[H]
	\centering
	\includegraphics[width=0.9\textwidth]{figure/chapter-4/page/mahasiswa-profile-radar.png}
	\caption{Halaman Profil Mahasiswa dengan Visualisasi Radar Chart Tipe Kepribadian MBTI}
	\label{fig:4.mahasiswa-profile-radar}
\end{figure}

Halaman profil juga menyediakan akses ke dialog modal yang menampilkan grid lengkap dari 16 tipe kepribadian MBTI, memungkinkan mahasiswa untuk mengeksplorasi deskripsi dan karakteristik dari berbagai tipe kepribadian. Setiap tipe ditampilkan dengan ikon karakter unik dan kategorinya (Analysts, Diplomats, Sentinels, Explorers).

\begin{figure}[H]
	\centering
	\includegraphics[width=0.9\textwidth]{figure/chapter-4/page/mahasiswa-mbti-distribution.png}
	\caption{Dialog Persebaran 16 Tipe Kepribadian MBTI dengan Tipe Mahasiswa yang Tersorot}
	\label{fig:4.mahasiswa-mbti-distribution-hover}
\end{figure}

